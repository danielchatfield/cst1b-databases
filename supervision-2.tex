\documentclass{supervision}
\usepackage{course}

\Supervision{2}
\begin{document}
  \begin{questions}
    \section*{2009 Paper 4 Question 8}
    \question
      \begin{parts}
        \part[2] Define the concept of a \emph{functional dependency}.
        \begin{solution}
          A set of attributes $X$ functional determines another set of
          attributes if, and only if, each $X$ value is associated with
          precisely one $Y$ value.
        \end{solution}

        \part Let \lstinline|R(A, B, C, D, E, F)| be a database schema with
          functional dependencies
          \begin{align*}
            A, B &\to C    \\
            B, C &\to A, D \\
               D &\to E    \\
            C, F &\to B
          \end{align*}
          \begin{subparts}
            \subpart[3] Compute the closure of $\{A,\, B\}$.
            \begin{solution}
              $\{ A, B, C, D, E \}$
            \end{solution}

            \subpart[1] Is $A, B \to D, F$ a functional dependency over $R$?
              Justify your answer.
              \begin{solution}
                No, because $F$ is not in the closure.
              \end{solution}
          \end{subparts}

        \part[2] Define the concept of a \emph{multivalued dependency}.
          \begin{solution}
            A multivalued dependency on $R$ is a constraint such that if two
            tuples of $R$ agree on all the attributes of $X$, then their
            components in $Y$ can be swapped and the result will be two tuples
            that are also in the relation.
          \end{solution}

        \part[3] Suppose the functional dependency $X \to Y$ holds on a
          relational schema. Does this mean that the multivalued dependency $X
          \twoheadrightarrow Y$ holds? Justify your answer.
          \begin{solution}
            Every functional dependency is a multivalued dependency since if it
            is a functional dependency then swapping their components in $y$
            will result in the same tuple (since they are neccessarily the same
            for tuples that agree on $X$).
          \end{solution}

        \part[3] Define the concept of a \emph{lossless-join decomposition}.
          \begin{solution}
            A lossless-join decomposition is splitting a relation $R$ into two
            relations $R_1$ and $R_2$ such that $R_1 \bowtie R_2 = R$.
          \end{solution}

        \part[6] Let $R(X)$ be a database schema, where $X$ is a set of
          attributes. Show that $S(Y \cup Z)$ and $T(Y \cup (X - Z))$ is a
          lossless-join decomposition of $R(X)$ if and only if the multivalued
          dependency $Y \twoheadrightarrow Z$ holds over $R$.
          \begin{solution}
            \begin{itemize}
              \item Suppose $Z \twoheadrightarrow W$
              \item We know (from proof of Heath's rule) that $R \subseteq
                \pi_{Z,W}(R) \bowtie \pi_{Z,Y}(R)$ so we only need to show
                $\pi_{Z,W}(R) \bowtie \pi_{Z,Y}(R) \subseteq R$.
              \item Suppose $r \in \pi_{Z,W}(R) \bowtie \pi_{Z,Y}(R)$.
              \item There must be a $t \in R$ and $u \in R$ with $\{ r \} =
                \pi_{Z,W}(\{ t \}) \bowtie \pi_{Z,Y}(\{ u \})$.
              \item In other words there must be a $t \in R$ and $u \in R$ with
                $t.Z = u.Z$
              \item So the MVD tells is that there must be some tuple $v \in R$
                such that
                \begin{itemize}
                  \item $v$ agrees with both $t$ and $u$ on the attributes of
                    $Z$
                  \item $v$ agrees with $t$ on the attributes of $W$.
                  \item $v$ agrees with $u$ on the attributes of $Y$.
                \end{itemize}
              \item This $v$ must be the same as $r$, so $r \in R$.
              \item Suppose $R = \pi_{Z,W}(R) \bowtie \pi_{Z,Y}(R)$
              \item Let $t$ and $u$ be any records in $R$ with $t.Z = u.Z$.
              \item Let $v$ be defined by $\{ v \} = \pi_{Z,W}(\{ t\}) \bowtie
                \pi_{Z,Y}(\{ u\})$
              \item By construction we have:
                \begin{itemize}
                  \item $v.Z = t.Z = u.Z$
                  \item $v.W = t.W$
                  \item $v.Y = u.Y$
                \end{itemize}
            \end{itemize}
            Therefore, $Z \twoheadrightarrow W$ holds.
          \end{solution}

      \end{parts}
    \section*{2010 Paper 4 Question 5}
    \question
      \begin{parts}
        \part Suppose that $R(A, B, C)$ is a relational schema with functional
          dependencies $F = \{ A,B \to C, C \to B\}$.
          \begin{subparts}
            \subpart[2] Is this schema in 3NF? Explain.
              \begin{solution}
                $A, B$ and $A, C$ are keys for $R$ and therefore there are no
                non-prime attributes.
              \end{solution}
            \subpart[2] Is this schema in BCNF? Explain.
              \begin{solution}
                $C$ is not a key and $C \to B$ therefore the schema is not in
                BCNF.
              \end{solution}

          \end{subparts}
        \part Decomposition plays an important role in database design.
          \begin{subparts}
            \subpart[2] Define what is meant by a \emph{lossless-join
              decomposition}.
                \begin{solution}
                  A lossless-join decomposition is splitting a relation $R$ into
                  two relations $R_1$ and $R_2$ such that $R_1 \bowtie R_2 = R$.
                \end{solution}
            \subpart[2] Define what is meant by a \emph{dependency preserving
              decomposition}.
              \begin{solution}
                A decomposition is dependency preserving if enforcing all
                functional dependencies on the two decomposed relations is the
                same as enforcing them on the join of them.
              \end{solution}
          \end{subparts}
        \part Let $R(A, B, C, D, E)$ be a relational schema with the following
          functional dependencies.
          \begin{align*}
            A, B &\to C \\
            D, E &\to C \\
               B &\to D \\l
          \end{align*}
          \begin{subparts}
            \subpart[2] What is the closure of $\{ A, B \}$?
              \begin{solution}
                $\{ A, B, C, D\}$
              \end{solution}
            \subpart[2] What is the closure of $\{ B, E \}$?
              \begin{solution}
                $\{ B, C, D, E\}$
              \end{solution}
            \subpart[8] Decompose the schema to BCNF in two different ways. In
              each case, are all dependencies preserved? Explain.
              \begin{solution}
                \begin{description}
                  \emph{Not sure about this}
                \end{description}
              \end{solution}
          \end{subparts}
      \end{parts}
  \end{questions}
\end{document}
