\documentclass{supervision}
\usepackage{course}

\Supervision{1}

\begin{document}
  \begin{questions}
    \section*{2003 Paper 5 Question 8}
    \question
      \begin{parts}
        \part
          \begin{subparts}
            \subpart[5] Define the operators in the core relational algebra.
              \begin{solution}
                \begin{description}
                  \item[Union] $R \cup S$ is the union of $R$ and $S$.
                  \item[Intersection] $R \cap S$ is the intersection of $R$ and
                    $S$.
                  \item[Difference] $R - S$ is the difference between $R$ and
                    $S$.
                  \item[Product] $R \times S$ is the product of $R$ and $S$.
                  \item[Selection] $\sigma _p(R)$ is a selection over $R$. $P$
                    is a propositional formula and the operator selects all
                    tuples in $R$ for which $p$ holds.
                  \item[Projection] $\pi _x(R)$ is a projection over $R$.$x$ is
                    a set of attribute names.
                \end{description}

              \end{solution}


            \subpart[4] Define the domain relational calculus.
              \begin{solution}
                $Q = \{ (A_1 = v_1,\, A_2 = v_2, \ldots,\, A_k = v_k) | P(v_1,\, v_2, \ldots,\, v_k)\}$
              \end{solution}

            \subpart[3] Show how the relational algebra can be encoded in the
              domain relational calculus.
              \begin{solution}
                \begin{itemize}
                  \item Selection
                    \begin{description}
                      \item[Relational Algebra]
                        $Q = \sigma_{A > 12}(R)$
                      \item[Domain Relational Calculus]
                        Encoded as:\\
                        $Q = \{\{(A, a), (B, b), (C, c)\} |
                        \{\{(A, a), (B, b), (C, c)\} \in R \land a > 12\}$
                    \end{description}
                  \item Projection
                    \begin{description}
                      \item[Relational Algebra]
                        $Q = \pi_{B,C}(R)$
                      \item[Domain Relational Calculus]
                        Encoded as: \\
                        $Q = \{\{(B, b), (C, c)\} |
                        \{\{(A, a), (B, b), (C, c)\} \in R \}$
                    \end{description}
                \end{itemize}
              \end{solution}

          \end{subparts}
        \part A \emph{constraint} can be expressed using relational algebra.
          For example, $R = \emptyset$ specifies the constraint that relation
          $R$ must be empty, and $(R \cup S) \subseteq T$ specifies that every
          tuple in the union of $R$ and $S$ must be in $T$.

          Consider the following schema.

          \begin{code}[mathescape]{sql}
            RockStar($\overline{name}$, address, gender, birthday)
            RockManager($\overline{managername}$, starname)
          \end{code}

          \begin{subparts}
            \subpart[1] Give a constraint to express that rock stars must be
              either male or female.
              \begin{solution}
                $\sigma _{gender \, != \, 'male'\: AND\: gender\, !=\,
                'female'}(RockManager) = \emptyset$
              \end{solution}

            \subpart[3] Give a constraint to express the referential integrity
              constraint between the \lstinline|RockStar| and
              \lstinline|RockManager| relations. (Note: \lstinline|starname|
              is intended to be a foreign key.)
              \begin{solution}
                $\pi _{starname}(RockManager) \subseteq \pi _{name}(RockStar)$
              \end{solution}

            \subpart[4] Give a constraint to express the functional dependency
              \lstinline|name|$\rightarrow$\lstinline|address| for the RockStar
              relation.
              \begin{solution}
                \emph{Not covered this yet and I don't understand it from slides alone}
              \end{solution}

          \end{subparts}
      \end{parts}
    \question
      \section*{2004 Paper 5 Question 8}
      Assume a simple movie database with the following schema. (You may assume
      that producers have a unique certification number, \lstinline|Cert|, that
      is also recorded in the \lstinline|Movie| relation as attribute
      \lstinline|prodC#|; and no two movies are produced with the same title.)

      \begin{code}[mathescape]{}
        Movie($\overline{title}$, year, length, prodC#)
        StarsIn(movieTitle, movieYear, starName)
        Producer(name, address, $\overline{cert}$)
        MovieStar($\overline{name}$, gender, birthdate)
      \end{code}

      \begin{parts}
        \part Write the following queries in SQL:
          \begin{subparts}
            \subpart[1] Who were the male stars in the film \textit{The Red
              Squirrel}?
              \begin{solution}
                \begin{code}{SQL}
                  (SELECT name
                   FROM MovieStar
                   WHERE gender='male')
                  INTERSECT
                  (SELECT starName
                   FROM StarsIn
                   WHERE movieTitle='The Red Squirrel')
                \end{code}
              \end{solution}

            \subpart[2] Which movies are longer than \textit{Titanic}?
              \begin{solution}
                \begin{code}{SQL}
                  SELECT *
                  FROM Movie
                  WHERE length >
                      (SELECT length
                       from Movie
                       WHERE title='Titanic')
                \end{code}
              \end{solution}

          \end{subparts}
        \part ${SQL}$ has a boolean-valued operator \lstinline|IN| such that
          the expression \lstinline|s IN R| is \lstinline|true| when $s$ is
          contained in the relation $R$ (assume for simplicity that $R$ is a
          single attribute relation and hence $s$ is a simple atomic value).

          Consider the following nested \lstinline|SQL| query that uses the
          \lstinline|IN| operator:

          \begin{code}{sql}
            SELECT name
            FROM   Producer
            WHERE  cert IN (SELECT prodC#
                            FROM   Movie
                            WHERE  title IN (SELECT movieTitle
                                             FROM   StarsIn
                                             WHERE  starName='Nancho Novo'));
          \end{code}

          \begin{subparts}
            \subpart[1] State concisely what this query is intended to mean.
              \begin{solution}
                This query returns the names of the producers of the movies
                that 'Nancho Novo' starred in.
              \end{solution}

            \subpart[2] Express this nested query as a single
              \lstinline|SELECT-FROM-WHERE| query.
              \begin{solution}
                \begin{code}{SQL}
                  SELECT DISTINCT Producer.name
                  FROM   StarsIn, Movie, Producer
                  WHERE  StarsIn.starName = 'nancho Novo',
                         Movie.title = StarsIn.movieTitle,
                         Producer.cert = Movie.prodC#
                \end{code}
              \end{solution}

            \subpart[6] Is your query from part (b)(ii) always equivalent to
              the original query? If yes, then justify your answer; if not,
              then explain the difference and show how they could be made
              equivalent.
              \begin{solution}
                The query is always equiavalent, the DISTINCT ensures that
                a producer is only returned once, just like the original query.

                The outer-most part of the original query was over the Producer
                table and therefore a single producer could only be returned
                once, even if they produced multiple films with Nancho Novo
                since the predicate IN is simply $true$ or $false$.

                The new query joins the tables together which duplicates the
                Producer field and therefore the \lstinline|DISTINCT| is
                required to ensure no duplicates.
              \end{solution}
          \end{subparts}

        \part[8] \lstinline|SQL| has a boolean-valued operator
          \lstinline|EXISTS| such that \lstinline|EXISTS R| is true if and only
          if $R$ is not empty.

          Show how \lstinline|EXISTS| is, in fact, redundant by giving a simple
          ${SQL}$ expression that is equivalent to \lstinline|EXISTS R| but
          does not involve \lstinline|EXISTS| or any cardinality operators,
          e.g. \lstinline|COUNT|. \emph{[Hint: You may use the \lstinline|IN|
          operator.]}
          \begin{solution}
            \emph{I think this is a valid solution but the number of marks
            available for this question is making me doubt that.}
            \begin{code}{sql}
              R IN R
            \end{code}

            An example:
            \begin{code}{sql}
              SELECT *
              FROM Colleges
              WHERE EXISTS (
                  SELECT *
                  FROM Students
                  WHERE Colleges.id = Students.college
              )
            \end{code}
            can be written as:
            \begin{code}{sql}
              SELECT *
              FROM Colleges
              WHERE id IN (
                  SELECT college
                  FROM Students
              )
            \end{code}
          \end{solution}
      \end{parts}


  \end{questions}
\end{document}
